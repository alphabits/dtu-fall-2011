\def\assignmenttitle{Modelling and Predicting the Number of Airline Passengers}
\def\assignmentnumber{3}
\def\assignmentdate{29-10-2011}

\documentclass[11pt]{article}
\linespread{1}

\renewcommand{\thefootnote}{\fnsymbol{footnote}}

\usepackage{geometry} % see geometry.pdf on how to lay out the page. There's lots.
\usepackage[utf8]{inputenc}
\usepackage{array}
\usepackage{amsmath,amssymb,latexsym,epic,eepic,epsfig,graphics,psfrag}
\usepackage{amsfonts}
\usepackage{graphicx,float}

\usepackage[danish]{babel}

\usepackage[bottom]{footmisc}

\usepackage{fancyhdr}
\pagestyle{fancy}
\lhead{\small\textit{01246 Partial Differential Equations - Fall 2011 - Anders Hørsted (s082382)}}
\rhead{\thepage}
\chead{}
\lfoot{}\cfoot{}\rfoot{}

\usepackage{pstricks}
\usepackage{pst-node}
\usepackage{wrapfig}
\usepackage{caption}
\usepackage{multirow}
%\usepackage{fouriernc}
%\usepackage[charter]{mathdesign}
\usepackage{lmodern}
\usepackage[normalem]{ulem}
\geometry{a4paper} % or letter or a5paper or ... etc
% \geometry{landscape} % rotated page geometry

\usepackage{subfigure}
\usepackage{placeins}
\usepackage{url}
\usepackage{natbib}
\renewcommand\bibsection*{}
\bibliographystyle{plain}

\makeatletter
\renewcommand*\env@matrix[1][*\c@MaxMatrixCols c]{%
  \hskip -\arraycolsep
  \let\@ifnextchar\new@ifnextchar
  \array{#1}}
\makeatother

\newcommand\myimp{\quad\Leftrightarrow\quad}
\newcommand\half{\frac{1}{2}}
%\newcommand\myvec[1]{\mathbf{#1}}
\newcommand\myvec[1]{\boldsymbol{#1}}
\newcommand\vecx{\myvec{x}}
\newcommand\mymod[1]{\ (\text{mod }#1)}
\newcommand\myreal{\mathbb{R}}
\newcommand\mynatural{\mathbb{N}}
\newcommand\myinteger{\mathbb{Z}}
\newcommand\mycomplex{\mathbb{C}}
\newcommand\myint{\text{int}}
\newcommand\norm[1]{||\,#1\,||}
\newcommand\bignorm[1]{\big|\big|\,#1\,\big|\big|}
\newcommand\seq[1]{\big\{#1\big\}}
\newcommand\smallseq[1]{\{#1\}}
\newcommand\smallseqtoinf[1]{\smallseq{#1}_{k=1}^\infty}
\newcommand\lonew{\ell^1_w}
\newcommand\lone{\ell^1}
\newcommand\ltwo{\ell^2(\mynatural)}
\newcommand\ip[2]{\langle#1,#2\rangle}
\newcommand\hilbert[1]{\mathcal{#1}}
\newcommand\uinf{u_{\infty}}
\newcommand\erf{\text{erf\,}}
\newcommand\infint{\int_{\infty}^{\infty}}
\newcommand\celsius{$^\circ$C}
\newcommand\comsol{Comsol}
\newcommand\fourier{\mathcal{F}}

\usepackage{tabulary}
\newcolumntype{y}{>{\centering\arraybackslash}R}

\setlength{\unitlength}{2mm}
\usepackage{tikz}

\title{Homework \homeworknumber}
\author{01246 Partial differential equations -- \homeworkdate -- Anders Hørsted (s082382)}
%\author{}
\date{} % delete this line to display the current date


%%% BEGIN DOCUMENT
\begin{document}

\maketitle

In this report a model of the monthly number of airline passengers in the U.S. is build. The data set used to build the model is the actual number of passengers for every month between January 1995 and March 2002. To be able to give an estimate of the precision of the model, the data set is separated into two parts: A training set containing data for the period January 1995 to June 2001, and a test set containing data for the period July 2001 to March 2002. Throughout the modelling phase we work as if only the training set is available. The test set is used when the final model have been build, to compare the predictions of the final model, and the actual numbers in the test set. From now and until the section about measuring the model performance the training set is just referred to as ``the data set'' or ``the data''

\section*{Data exploration}
In this section the data set is introduced. First a plot of the data is created and shown in figure~\ref{fig:trainingset}. \par
\begin{figure}[ht]
\centering
\includegraphics[width=120mm]{../plots/trainingset.pdf}
\caption{Plot of data set used for modelling}
\label{fig:trainingset}
\end{figure}
From the plot a general upward trend is recognized. This isn't surprising since the U.S. economy got stronger (REFERENCE!!!) in the period and therefore more airline passengers should be expected. Also a regular seasonal pattern can be seen which isn't surprising either. In the summer months e.g. we would expect more passengers than for the other months etc. The conclusion of these two observations is that the time series is non-stationary and this is something that should be coped with during the modelling phase. To support that the time series is non-stationary the estimated autocorrelation function and partial autocorrelation function are now plotted (see figure~\ref{fig:acfs-trainingset})
\begin{figure}
    \centering
    \mbox{  
        \subfigure{\includegraphics[width=70mm]{../plots/acf-trainingset.pdf}} \quad 
        \subfigure{\includegraphics[width=70mm]{../plots/pacf-trainingset.pdf}} 
    }
    \caption{CAPTION!!!}
    \label{fig:label}
\end{figure}



\pagebreak
\section*{Appendices}
All R code used for this assignment is included here. All source code incl. latex code for this report can be found at {\small\url{https://github.com/alphabits/dtu-fall-2011/tree/master/02417/assignment-3}}

\pagebreak

\begin{thebibliography}{9}

\bibitem{nocedal06}
  Jorge Nocedal \& Stephen J. Wright,
  \emph{Numerical Optimization}.
  Springer Science+Business Media,
  2nd Edition,
  2006.
\bibitem{nielsen10}
  Kaj Madsen \& Hans Bruun Nielsen,
  \emph{Introduction to Optimization and Data Fitting}.
  DTU IMM,
  1st Edition,
  2010.

\end{thebibliography}


\end{document}
